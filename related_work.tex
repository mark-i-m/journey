\documentclass[twocolumn,11pt]{article}

\usepackage[margin=1in]{geometry}
\usepackage{caption}

\title{Journey: A Study of Studying Large Memory Systems}
\author{Mark Mansi, Suhas Pai, Hasnain Ali Pirzada}
\date{}

\begin{document}

\maketitle

The emergence of new technology has in the past often required operating
systems to change. Traditionally, OS researchers and developers benchmark and
test their solutions in new environments to demonstrate their efficacy before
using them in real systems. However, researchers do not always have access to
new technology for practical reasons; for example, large memory systems are
still too expensive for most research budgets. This is not the first time that
researchers have been forced to study systems in environments they do not have
access to. This section examines past approaches to studying these systems and
motivates the need for our work.

The emergence of the internet required businesses to buy expensive machines to
keep up with demand for their services, but systems researchers did not have
access to these machines. Alameldeen et al. describe how they simulate a
multi-million dollar server using a \$2000 workstation. To do this, they go
through several rounds of scaling down, optimizing, changing, and tuning both
the benchmarks and systems they test. For example, they scale down a workload
to fit in the 1GB memory of the machine they use and increase the number of
threads to improve parallelism (TODO: cite). This methodology may be accurate,
but it requires modifying the benchmarks, which is error prone and time
consuming; it can be extremely difficult to validate changes to a benchmark. In
our work, we seek to avoid changes to the experimental workloads including
changes to code or scale. 

Likewise, the Quartz emulator studies the behaviour of applications on
Non-volatile Memory (NVM) while actually running on top of DRAM. It uses
existing hardware capabilities to emulate the higher latency and lower
bandwidth of most NVM technologies. While Quartz does not address the fact that
the actual memory available to the applications on NVM might be orders of
magnitude higher than on DRAM, it demonstrates that new technologies may be
emulated using existing technology (TODO: cite [2]). 

The Simics simulator uses demand paging for simulated environments, which means
that memory is only allocated when it is used. As long as the working set of
the target system fits in the host memory, performance will be tolerable.
However, in our work, we wish to study OS behavior under workloads that
actually use all available memory. Thus, the usefulness of demand paging is
greatly reduced. For these workloads, the performance of simulators like Simics
tends to be impractical (TODO: cite [3]). We do not explore simulation further
because it can be orders of magnitude slower than native execution, making it
infeasible for studying large workloads or systems (TODO cite: \$2M paper).

A similar problem was encountered by researchers studying both large storage
drives and large distributed storage systems. David is a system that allows
storage and big data researchers to run large benchmarks requiring terabytes of
storage using off the shelf storage devices (which at that time were too
small). David creates a compressed version of the file system by physically
storing only metadata and discarding the contents of files. Reads to the disk
causes David to generate data on the fly. This decision choice is based on the
observation that most benchmarking frameworks do not care about the actual
content of the files, and that most of the storage capacity of a drive tends to
be data rather than metadata (TODO: cite [1]). The Exalt system uses a similar
methodology for large-scale distributed storage systems (TODO: cite [4]). This
methodology provides a promising direction for large-memory system studies. One
may consider ignoring the contents of a process’s heap and only storing kernel
data structures and a process’s code and stack segments. However, generating
heap contents on the fly is more difficult than generating disk contents
because of the common use of custom data structures. Moreover, on Linux, the
kernel data structures for 1TB of memory may also exceed the size of physical
memory on present systems[3].

A virtualized environment can be used to provide a guest system with more
memory than is physically available to the host. A study of the limits of the
KVM hypervisor found that there is no fundamental limit to the size of guest
physical memory other than the hardware address width. However, currently, a
Linux host running KVM will require the guest memory to be backed by host
physical memory or host swap space (TODO: cite IBM KVM).  This means that when
the virtual machine uses the whole amount of memory allocated to it, the host
can swap pages to disk. The resulting poor performance can cause inaccurate
performance measurements when running benchmarks.

Prefetching pages from swap space can offer a way to mitigate the overhead of
memory overcommitment. When there is significant memory pressure, even pages
which are likely to be accessed soon are swapped out to disk. They are faulted
back into memory when accessed, resulting in significant performance cost.
Charm++ uses a programming model where computations can be scheduled by the
language runtime. A designated thread can prefetch pages required by a
computation, averting page faults. Although this approach is effective in
avoiding overhead, it requires applications to be written with a particular
programming language and so is not applicable to most benchmarks without
changing them (TODO: cite Charm++ paper).

Gupta et al. built an emulator for high speed networks using a technique called
time dilation, which slows down the OS clock to make it appear that external
events are occurring faster. This allows the system to emulate network links
with speeds that are currently not available. The implementation is based on
the VMs and Xen hypervisor; Xen delivers the timer interrupts to the guest at a
lower rate than hardware hence  slowing down the guest’s clock (TODO: cite). We
may use a similar approach to slow down system time  while paging in and out
large portions of memory. This will allow us to make the system believe that it
is reading data from the memory while actually most of that data is being read
from the disk.

Finally, there have been studies which look at the performance overheads
associated with current implementations of virtual memory and suggest
mechanisms to mitigate them. RadixVM tries to overcome performance issues in
highly concurrent workloads due to serialization of memory management
operations on kernel data structures (TODO: cite RadixVM). This work
demonstrates that many parts of existing memory management schemes are not
scalable to larger systems. Similarly, in our work, we wish to examine
scalability limitations of memory management in the Linux kernel. Other studies
have suggested that \texttt{struct page}, \texttt{struct vm\_area\_struct}, and page tables
tend to comprise a large portion of memory management overhead (TODO: cite
Simics article).

\bibliography{references}{}
\bibliographystyle{plain}

\end{document}
